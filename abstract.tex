On the diverse areas of human knowledge, there are a series of well-known optimization
problems. To solve these problems consists on finding, in a collection of configurations,
a subset, or even an element that better satisfies one or more constraints previously determined. A
famous strategy to solve such problems in a quick and efficient manner consists on the use of
genetic algorithms. They receive such name due to the strong inspiration in phenomena of
evolutionary biology\trav like mutation, recombination (or crossover) and selection\trav in the
elaboration of its steps of execution. In this work, a genetic algorithm is proposed for the
optimization of real functions, that is, for the search of the points in which the function is
maximum (or minimum). Its implementation is done using the Python programming language, with the help of the NumPy library.
The algorithm will be applied in the optimization of some example functions, and a discussion
is made concerning the results, as well as the efficiency and performance of the implementation.

\vspace{\onelineskip}\noindent
\textbf{Keywords}: Genetic Algorithm. Numerical Optimization. Python. NumPy.
