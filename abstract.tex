On the diverse areas of human knowledge, there are a series of well-known optimization
problems. The solution of these problems consists in finding, in a collection of configurations,
a subset, or even an element that better satisfies one or more constraints previously determined. A
famous strategy to solve such problems in a quick and efficient fashion relies on the use of
genetic algorithms. They receive such name due to the strong inspiration in phenomena of
evolutionary biology\trav like mutation, recombination (or crossover) and selection\trav in its execution. 
In this project, a genetic algorithm is proposed for the
optimization of real functions, that is, for the search of the points in which the function is
maximum (or minimum). Its implementation is done using the Python programming language, by means of the NumPy library.
The code developed was applied in the optimization of some example functions, and a discussion
was conducted concerning the results, as well as on the efficiency and performance of the implementation.

\vspace{\onelineskip}\noindent
\textbf{Keywords}: Genetic Algorithm, Numerical Optimization, Maximum of Functions, Minimum of Functions, Python.
