\begin{figure}
  \centering
  \begin{tikzpicture}
    \node[start_end] (start) {Início};
    \node[io, below=1em of start] (init_pop) {População\\ Inicial};
    \node[process, below=1em of init_pop]  (selection) {Seleção};
    \node[process, below=1em of selection]  (crossover) {Recombinação};
    \node[process, below=1em of crossover]  (mutation) {Mutação};
    \node[decision, below=1em of mutation]  (check) {Condição atingida?};
    \node[document, left=10ex of check] (print_stats) {Informações da\\População};
    \node[start_end, above=10ex of print_stats] (end) {Fim};
    \draw[myarrow=.9] (start.south) --  (init_pop.north);
    \draw[myarrow=.9] (init_pop.south) --  (selection.north);
    \draw[myarrow=.9] (selection.south) --  (crossover.north);
    \draw[myarrow=.9] (crossover.south) --  (mutation.north);
    \draw[myarrow=.9] (mutation.south) -- (check.north);
    \draw[myarrow=.9] (check.east) -- node[description, above] {Não} ([xshift=10ex]check.east) |- (selection.east);
    \draw[myarrow=.9] (check.west) -- node[description, above] {Sim} (print_stats.east);
    \draw[myarrow=.9] (print_stats.north) -- (end.south);
  \end{tikzpicture}
  \caption{Fluxograma geral de um algoritmo genético.}
  \label{fig:ga_flow}
\end{figure}