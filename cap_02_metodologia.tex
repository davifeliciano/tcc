\chapter{Metodologia}\label{cap_metodologia}

\section{Codificação}

\newcommand{\kth}[2]{{#1}^{(k)}_{#2}}
\newcommand{\kpth}[2]{{#1}^{(k + 1)}_{#2}}
\newcommand{\xmin}[1]{x^{(min)}_{#1}}
\newcommand{\xmax}[1]{x^{(max)}_{#1}}

Seja uma população de $n$ indivíduos representada pelo conjunto
\begin{equation}
  A = \left\{ A_1\;, \mathdots,  \;A_k\;, \mathdots, \;A_n \right\} \mathcomma
\end{equation}
onde cada indivíduo tem seu código genético representado por $m$ cromossomos
de $l$ bits, de forma que, usando uma notação matricial,
\begin{equation}
  A_k = \left[
    \begin{matrix}
      \kth{a}{11} & \cdots & \kth{a}{1m} \\
      \vdots      & \ddots & \vdots      \\
      \kth{a}{l1} & \cdots & \kth{a}{lm} \\
    \end{matrix}
    \right]
  \mathcomma
\end{equation}
onde $ \kth{a}{ij} \in \{0,1\} $, para $i$, $j$ e $k$ inteiros tais que
$ i \in \left[ 0, l \right] $, $ j \in \left[ 0, m \right] $ e $ k \in \left[ 0, n \right] $.

Para um problema de otimização numérica de uma função $ f : \mathcal{C} \subseteq \R^m \rightarrow \R $,
deve existir um mapa injetivo entre $A$ e o conjunto
\begin{equation}
  X = \left\{ X_1 \; , \mathdots,  \; X_k \; , \mathdots, \; X_n \right\} \mathcomma
\end{equation}
onde
\begin{equation}
  X_k = \left( \kth{x}{1}, \mathdots, \; \kth{x}{j}, \mathdots, \; \kth{x}{m} \right)
\end{equation}
são vetores de $ \R^m $ e representam a posição do respectivo indivíduo no espaço de busca.
Tal espaço é um subconjunto de $ \mathcal{C} $ definido pelo produto cartesiano
\begin{equation}
  \mathcal{S} = \bigtimes_{j = 1}^m \; [ \xmin{j} \; , \; \xmax{j} ] \mathcomma
\end{equation}
sendo $ \xmin{j} $ e $ \xmax{j} $, respectivamente, os valores mínimo e máximo para
a coordenada $ \kth{x}{j} $.

Como $l$ bits são capazes de representar números inteiros em $ [0, 2^l - 1) $,
uma forma natural de mapear tal intervalo
em $ [ \xmin{j} \; , \; \xmax{j} ] $ é
\begin{equation}
  x^{(k)}_j = \xmin{j} + \frac{\xmax{j} - \xmin{j}}{2^l - 1} \sum_{i = 1}^l \kth{a}{ij} 2^{i - 1} \mathcomma
\end{equation}
cuja relação inversa é determinada por
\begin{equation}
  \sum_{i = 1}^l \kth{a}{ij} 2^{i - 1} =
  \left\lfloor \frac{(\kth{x}{j} - \xmin{j})(2^l - 1)}{\xmax{j} - \xmin{j}} \right\rfloor
  \mathperiod
\end{equation}

\section{Seleção}

Para o processo de seleção é introduzida uma nova função, denominada função desempenho.
Tal função é alguma função $ g : \R \rightarrow \R $ a partir da qual se calcula a probabilidade
de seleção do indivíduo $k$
\begin{equation}
  P_k = \frac{g(f(X_k))}{\sum_{j = 1}^n g(f(X_j))} \mathperiod
\end{equation}

Existe uma liberdade para a escolha da função desempenho, desde que se respeite a condição a
função seja sempre positiva. A ideia é que se escolha uma função desempenho que
selecione os melhores indivíduos, mas ainda mantenha uma boa variedade genética na
próxima geração. Assim, em toda geração, em média, haverá indivíduos em todo o espaço
de busca, evitando que a solução convirja de forma prematura\footnote{
  Retomando a função usada como exemplo no capítulo anterior, definida na Equação
  \paref{eq:damped_cos}, se escolhermos uma função de desempenho que privilegie
  de forma desproporcional pontos onde o valor da função é maior. Dependendo da
  distribuição inicial dos indivíduos, é possível que após algumas gerações,
  toda a população se concentre em $ r = \nicefrac{1}{2n} $, que é um
  máximo local.
}.

\newcommand{\fmin}{f_{min}}
\newcommand{\fmax}{f_{max}}

Nesse trabalho a função desempenho usada foi um escalamento linear\cite{goldberg1989ga} $ g(x) = ax + b $
sendo $a$ e $b$ definidos por
\begin{equation}
  a =
  \begin{cases}
    \frac{\mu (h - 1)}{\fmax - \mu} \;, & \text{se } \fmin > \frac{h\mu - \fmax}{h - 1} \\
    \frac{\mu}{\mu - \fmin}         \;, & \text{caso contrário}
  \end{cases}
  \label{eq:linear_fit_a}
\end{equation}
e
\begin{equation}
  b =
  \begin{cases}
    \frac{\mu (\fmax - h\mu)}{\fmax - \mu} \;, & \text{se } \fmin > \frac{h\mu - \fmax}{h - 1} \\
    - \frac{\mu\fmin}{\mu - \fmin}         \;, & \text{caso contrário}
  \end{cases}
  \mathcomma
  \label{eq:linear_fit_b}
\end{equation}
onde
\begin{equation}
  \fmin = \min_{X_k \in X} f(X_k) \mathcomma
\end{equation}
\begin{equation}
  \fmax = \max_{X_k \in X} f(X_k) \mathcomma
\end{equation}
$h$ é um parâmetro real maior que $1$ e $\mu$ é a média do valor de $f$ sobre a população, dada por
\begin{equation}
  \mu = \sum_{k = 1}^n \frac{f(X_k)}{n} \mathperiod
\end{equation}

A seleção do indivíduos que serão recombinados para formar a geração seguinte é feita por meio
de uma roleta simples, permitindo ou não repetições. São feitos $ \nicefrac{n}{2} $ sorteios,
já que para cada par de indivíduos recombinados, dois novos serão gerados\footnote{
  Por esse motivo, $n$ deve ser um múltiplo de 4.
}.
Desse modo, a geração subsequente será constituída pelos indivíduos pais selecionados e os seus
respectivos filhos, seguindo com $n$ indivíduos.

\section{Recombinação e Mutação}

Seja o conjunto de indivíduos selecionados
\begin{equation}
  S = \left\{ S_1\;, \mathdots,  \;S_k\;, \mathdots, \;S_{\nicefrac{n}{2}} \right\} \mathperiod
  \label{eq:selection}
\end{equation}
Estes indivíduos passarão, em pares adjacentes\footnote{
  O par $ (S_1, S_2) $ dá origem a dois indivíduos. O par $ (S_3, S_4) $ forma outros dois, e assim segue
  até que $ \nicefrac{n}{2} $ novos indivíduos sejam gerados.
}, pelo processo de recombinação e mutação, para dar origem a
geração seguinte da população. Esse processo é iniciado com o sorteio de duas posições em cada cromossomo,
para cada um desses pares. Essas posições são chamadas de pontos de recombinação.

Em seguida, o material genético dos indivíduos filhos será obtido pela união das partes complementares
de cada par de cromossomos dos pais e, por fim, são introduzidas mutações nos genes, com uma probabilidade predefinida.

\begin{figure}
  \centering
  \resizebox*{\textwidth}{!}{
    \begin{tikzpicture}
      \def\verticalmargin{1.3}
      \begin{scope}[local bounding box=fig1]
        \begin{scope}[start chain=1 going right,node distance=-0.15mm]
          \node [on chain=1,gene] (parent_1_left)     {0};
          \node [on chain=1,gene] (recomb_point_1_up) {0};
          \node [on chain=1,gene_filled]              {1};
          \node [on chain=1,gene_filled]              {0};
          \node [on chain=1,gene_filled]              {1};
          \node [on chain=1,gene_filled]              {1};
          \node [on chain=1,gene] (recomb_point_2_up) {0};
          \node [on chain=1,gene] (parent_1_right)    {1};
          \node [left=of parent_1_left] {$S_k$};
        \end{scope}
        \begin{scope}[start chain=1 going right,node distance=-0.15mm, shift={(0,-\verticalmargin)}]
          \node [on chain=1,gene] (mask_left)  {0};
          \node [on chain=1,gene]              {0};
          \node [on chain=1,gene]              {1};
          \node [on chain=1,gene]              {1};
          \node [on chain=1,gene]              {1};
          \node [on chain=1,gene]              {1};
          \node [on chain=1,gene]              {0};
          \node [on chain=1,gene] (mask_right) {0};
          \node [left=of mask_left] {$R$};
        \end{scope}
        \begin{scope}[start chain=1 going right,node distance=-0.15mm, shift={(0,-2*\verticalmargin)}]
          \node [on chain=1,gene_filled] (parent_2_left)       {1};
          \node [on chain=1,gene_filled] (recomb_point_1_down) {1};
          \node [on chain=1,gene]                              {1};
          \node [on chain=1,gene]                              {1};
          \node [on chain=1,gene]                              {1};
          \node [on chain=1,gene]                              {0};
          \node [on chain=1,gene_filled] (recomb_point_2_down) {1};
          \node [on chain=1,gene_filled] (parent_2_right)      {0};
          \node [left=of parent_2_left] {$S_{k + 1}$};
        \end{scope}
        \begin{scope}[start chain=1 going right,node distance=-0.15mm, shift={(0,-3*\verticalmargin)}]
          \node [on chain=1,gene_mutated] (mutation_left) {1};
          \node [on chain=1,gene]                         {0};
          \node [on chain=1,gene]                         {0};
          \node [on chain=1,gene]                         {0};
          \node [on chain=1,gene]                         {0};
          \node [on chain=1,gene]                         {0};
          \node [on chain=1,gene]                         {0};
          \node [on chain=1,gene]                         {0};
          \node [left=of mutation_left] {$M$};
        \end{scope}
        \draw ($(fig1.north west)+(0,0.85)$) rectangle ($(fig1.south east)+(0.15,-0.15)$);
        \draw [densely dotted] (recomb_point_1_up.east) -- (recomb_point_1_down.east);
        \draw [densely dotted] (recomb_point_2_up.west) -- (recomb_point_2_down.west);
        \node [description, font=\tiny, above=1.5ex of recomb_point_1_up.east] {Ponto de\\ Recombinação};
      \end{scope}
      \begin{scope}[xshift=21em, local bounding box=fig2]
        \begin{scope}[start chain=1 going right,node distance=-0.15mm]
          \node [on chain=1,gene_mutated] (mut_offspring_1_left) {1};
          \node [on chain=1,gene]                                {0};
          \node [on chain=1,gene]                                {1};
          \node [on chain=1,gene]                                {1};
          \node [on chain=1,gene]                                {1};
          \node [on chain=1,gene]                                {0};
          \node [on chain=1,gene]                                {0};
          \node [on chain=1,gene] (mut_offspring_1_right)        {1};
        \end{scope}
        \begin{scope}[start chain=1 going right,node distance=-0.15mm, shift={(0,-1*\verticalmargin)}]
          \node [on chain=1,gene] (offspring_1_left)  {0};
          \node [on chain=1,gene]                     {0};
          \node [on chain=1,gene]                     {1};
          \node [on chain=1,gene]                     {1};
          \node [on chain=1,gene]                     {1};
          \node [on chain=1,gene]                     {0};
          \node [on chain=1,gene]                     {0};
          \node [on chain=1,gene] (offspring_1_right) {1};
        \end{scope}
        \begin{scope}[start chain=1 going right,node distance=-0.15mm, shift={(0,-2*\verticalmargin)}]
          \node [on chain=1,gene_filled] (offspring_2_left)  {1};
          \node [on chain=1,gene_filled]                     {1};
          \node [on chain=1,gene_filled]                     {1};
          \node [on chain=1,gene_filled]                     {0};
          \node [on chain=1,gene_filled]                     {1};
          \node [on chain=1,gene_filled]                     {1};
          \node [on chain=1,gene_filled]                     {1};
          \node [on chain=1,gene_filled] (offspring_2_right) {0};
        \end{scope}
        \begin{scope}[start chain=1 going right,node distance=-0.15mm, shift={(0,-3*\verticalmargin)}]
          \node [on chain=1,gene_mutated] (mut_offspring_2_left) {0};
          \node [on chain=1,gene_filled]                         {1};
          \node [on chain=1,gene_filled]                         {1};
          \node [on chain=1,gene_filled]                         {0};
          \node [on chain=1,gene_filled]                         {1};
          \node [on chain=1,gene_filled]                         {1};
          \node [on chain=1,gene_filled]                         {1};
          \node [on chain=1,gene_filled] (mut_offspring_2_right) {0};
        \end{scope}
        \path [font=\tiny] (mask_right.east)++(0.15,0) edge node [above] {$(S_k \land R ) \lor (S_{k + 1} \land \lnot R)$} (offspring_1_left.west);
        \path [font=\tiny] (parent_2_right.east)++(0.15,0) edge node [below] {$(S_k \land \lnot R ) \lor (S_{k + 1} \land R)$} (offspring_2_left.west);
        \path [font=\tiny] (offspring_1_right.east) edge[bend right=90] node [left] {$\oplus M$} (mut_offspring_1_right.east);
        \path [font=\tiny] (offspring_2_right.east) edge[bend left=90] node [left] {$\oplus M$} (mut_offspring_2_right.east);
      \end{scope}
    \end{tikzpicture}
  }
  \caption{
    Diagrama ilustrando\trav em um caso com $m = 1$ e $l = 8$\trav a recombinação entre os indivíduos $S_k$ e $S_{k + 1}$
    e mutação nos filhos gerados segundo as respectivas máscaras de recombinação e mutação $R$ e $M$.
  }
  \label{fig:recomb_diagram}
\end{figure}


Matematicamente, podemos realizar esse processo para cada par da seguinte forma. Primeiro, devemos
sortear os pontos de recombinação. Esse processo é feito por meio da geração de $m$ pares
de números naturais aleatórios distintos $ \kth{\alpha}{j} $ e $ \kth{\beta}{j} $ tais que
$ 0 \leq \kth{\alpha}{j} < \kth{\beta}{j} $ e $ \kth{\alpha}{j} \leq \kth{\beta}{j} < l $ onde
$ j = 0, 1, \dots, m $ e $ k = 0, 2, 4, \dots, \nicefrac{n}{2} - 2 $.
Depois, é montada uma matriz $R$ denominada máscara de recombinação, definida por
\begin{equation}
  \kth{r}{ij} =
  \begin{cases}
    0 \;, & \text{se } i \leq \kth{\alpha}{j} \text{ ou } i > \kth{\beta}{j} \\
    1 \;, & \text{caso contrário}
  \end{cases}
  \mathcomma
\end{equation}
bem como uma máscara de mutação $M$, tal que
\begin{equation}
  \kth{m}{ij} =
  \begin{cases}
    1 \;, & \text{se houver mutação} \\
    0 \;, & \text{caso contrário}
  \end{cases}
  \mathperiod
\end{equation}

Finalmente, o conjunto dos indivíduos resultantes da recombinação da seleção no conjunto definido
pela Equação \paref{eq:selection} será
\begin{equation}
  S' = \left\{ S'_1\;, \mathdots,  \;S'_k\;, \mathdots, \;S'_{\nicefrac{n}{2}} \right\}
\end{equation}
tal que
\begin{equation}
  \kth{s'}{ij} = (\kth{s}{ij} \land \kth{r}{ij}) \lor (\kpth{s}{ij} \land \lnot \kth{r}{ij}) \oplus \kth{m}{ij}
\end{equation}
e
\begin{equation}
  \kpth{s'}{ij} = (\kth{s}{ij} \land \lnot \kth{r}{ij}) \lor (\kpth{s}{ij} \land \kth{r}{ij}) \oplus \kth{m}{ij} \mathcomma
\end{equation}
onde $\lnot$, $\land$, $\lor$ e $\oplus$ são os operadores lógicos de negação, conjunção, disjunção e disjunção exclusiva,
respectivamente.

Ao final desse processo\trav resumido no diagrama presente na Figura \paref{fig:recomb_diagram}\trav
teremos uma nova população, sobre a qual poderemos ordenar os indivíduos segundo a função desempenho, verificar
as características dos primeiros, e, caso necessário, repetir o algoritmo desde a etapa de seleção.

\section{Estratégia Elitista}

Afim de garantir a reprodução das características dos melhores indivíduos na geração seguinte, uma
estratégia elitista pode ser tomada\cite{goldberg1989ga}\cite{roncaratti2006ga}.
Seja $S_{\epsilon}$ o melhor indivíduo\footnote{
  Aquele cuja solução correspondente $X_{\epsilon}$ satisfaz $ g(f(X_{\epsilon})) = \max_{X_k \in X} g(f(X_k)) $.
}
do conjunto $S$ definido na Equação \paref{eq:selection}. A ideia dessa abordagem é introduzir\trav
antes do processo se recombinação\trav cópias de $S_{\epsilon}$ na população\footnote{
  O conjunto dessas cópias é chamado de elite da população.
}
com o objetivo de preservar seu material genético, assim como recombina-lo com o de outros indivíduos, na esperança
de obter soluções ainda melhores para o problema.

Para isso, substituímos o conjunto $S$ por
\begin{equation}
  S =
  \left\{
  \underbrace{S_{\epsilon}\;, \mathdots,  \;S_{\epsilon}}_{e_1 \text{ cópias}}\;,\;
  \overbrace{S_{\epsilon}\;, \mathdots,  \;S_{\epsilon}}^{e_2 \text{ cópias}}\;,\;
  \underbrace{S_{e_1 + e_2 + 1}\;, \mathdots,  \;S_{\nicefrac{n}{2}}}_{\mathllap{e_3 \text{ cópias dentre os restantes}}}
  \right\}
  \mathcomma
\end{equation}
onde os primeiros $e_1 + e_2$ indivíduos foram trocados por $S_{\epsilon}$\footnote{
  Como a recombinação ocorre em pares, $e_1$ e $e_2$ devem ser pares.
}, e, dentre os indivíduos restantes,
são feitas $e_3$ substituições similares em posições aleatórias.

Ao final desse processo, prossegue-se para a etapa de recombinação, com o detalhe de que os primeiros $e_1$
indivíduos terão probabilidade de mutação $p_1 = 0$, os $e_2$ indivíduos seguintes e o restante da população
terão probabilidades de mutação não nulas $p_2$ e $p_3$.
Como consequência, $S_{\epsilon}$ será parte da população seguinte, uma vez que a recombinação entre um
indivíduo e seu clone dá origem a dois outros clones, na ausência de mutações. Ademais, cópias mutadas
também estarão presentes, bem como cruzamentos de $S_{\epsilon}$ com $e_3$ indivíduos distintos.