\chapter{Conclusão e Perspectivas}\label{cap_conclusao}

Um algoritmo do tipo genético se mostrou eficaz na solução de problemas de otimização
numérica, encontrando o máximo global com poucas iterações, sendo uma
estratégia muito superior a qualquer método do tipo escalada de morro. 

Como pode ser observado, em todos os casos, os melhores indivíduos foram localizados no máximo global
da função, bastando a escolha correta dos parâmetros para que a solução desejada
fosse encontrada com precisão. Não obstante, uma variedade genética proporcional a
probabilidade de mutação foi mantida, dada uma escolha correta de $p_2$ e $p_3$ para cada
problema. Nos casos em que $p_2 = 20\%$, os indivíduos, ao final do processo, se encontravam
espalhados por todo o espaço de busca. 

Vale ressaltar porém que o valor necessário de $p_2$ e $p_3$ para que uma população 
tenha a distribuição desejada depende da função a ser otimizada, como pode ser visto
comparando as Figuras \paref{fig:contour_damped_cossine_mut_20} e \paref{fig:evolution_damped_cossine_mut_20}
com as Figuras \paref{fig:contour_near_gaussians_mut_20} e \paref{fig:evolution_near_gaussians_mut_20}.
Assim, a influência dos parâmetros $p_2$, $p_3$, $e_1$, $e_2$ e $e_3$ no comportamento da população
deve ser estudada em cada caso, afim de extrair do algoritmo o resultado desejado.

Outra vantagem do algoritmo é que os máximos locais também puderam ser encontrados.
Isso pode ser observado especialmente na Figura \paref{fig:evolution_damped_cossine_mut_20},
onde visivelmente há uma concentração de indivíduos em $ f_1(x,y) \approx 0,74 $ e 
$ f_1(x,y) \approx 0,29 $, que correspondem aos dois primeiros máximos locais.
Algo similar pode ser observado na segunda função, nas primeiras gerações da Figura 
\paref{fig:evolution_near_gaussians_mut_20}, com alguma concentração da população em
vermelho em $f_1(x,y) \approx 0,8$.

Claro que, mesmo nos casos em que não é possível inferir o máximo local da evolução
de uma parcela da população\trav como ocorre na Figura \paref{fig:evolution_damped_cossine}\trav
uma análise estatística feita sobre os valores das funções desempenho dos indivíduos e
suas respectivas posições ainda seria capaz de acusá-lo.

Em alguns casos, 100 gerações não são o suficiente para determinar a solução correta. Na população
em vermelho na Figura \paref{fig:evolution_near_gaussians} houve uma convergência rápida para o
máximo local, onde todos os 200 melhores indivíduos permaneceram durante toda a duração do teste.
Algo similar ocorreu na população em verde, para a maioria dos indivíduos da elite. Entretanto, 
o número de gerações sempre pode ser escolhido conforme a necessidade.

Em problemas mais complexos, ou que demandem maior precisão, o espaço de busca pode ser restrito
para uma região menor, contendo os candidatos a solução obtidos anteriormente. Na falta de
candidatos, o oposto pode ser feito, expandindo a região de busca. 

Uma outra característica 
do código implementado é a facilidade de paraleliza-lo. Em todos os resultados exibidos no capítulo
anterior, a evolução de cada população foi computada de forma concorrente em 8 processos distintos,
aumentando assim a amostra de indivíduos sem acréscimo significativo no tempo de execução.

Em suma, o algoritmo desenvolvido se mostrou bastante versátil, e produziu resultados favoráveis nos 
testes apresentados, sendo a maior dificuldade atrelada a solução do problema a determinação dos
parâmetros que melhor possibilitam o sucesso na busca dos extremos. As capacidades e limitações serão postos
a prova em outros testes subsequentes e nas futuras etapas do trabalho, onde será feita sua aplicação 
no estudo acerca de algum sistema físico.

\nocite{ribeiro2013ga}