\chapter{Conclusão}
\label{cap_conclusao}

Um algoritmo do tipo genético se mostrou eficaz na solução de problemas de otimização
numérica, encontrando o máximo global com poucas iterações, sendo uma
estratégia muito superior a qualquer método do tipo escalada de morro na
tarefa de mapear não só os máximos locais, mas também os máximos globais. 

Em problemas mais complexos, ou que demandem maior precisão, o espaço de busca pode ser restrito
para uma região menor, contendo os candidatos a solução obtidos anteriormente. Na falta de
candidatos, o oposto pode ser feito, expandindo a região de busca. 

Uma outra característica 
do código implementado é a facilidade de paraleliza-lo. Em todos os resultados exibidos no capítulo
anterior, a evolução de cada população foi computada de forma concorrente em 8 processos distintos,
aumentando assim a amostra de indivíduos sem acréscimo significativo no tempo de execução.

Em suma, o algoritmo desenvolvido se mostrou bastante versátil, e produziu resultados favoráveis nos 
testes apresentados, sendo a maior dificuldade atrelada a solução do problema a determinação dos
parâmetros que melhor possibilitam o sucesso na busca dos extremos. As capacidades e limitações serão postos
a prova em outros testes subsequentes e nas futuras etapas do trabalho, onde será feita sua aplicação 
no estudo acerca de algum sistema físico.

\nocite{ribeiro2013ga}