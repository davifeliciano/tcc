\chapter{Conclusão e Perspectivas}\label{cap_conclusao}

Um algoritmo do tipo genético se mostrou eficaz na solução de problemas de otimização
numérica, encontrando o máximo global com poucas iterações e mantendo uma varia, sendo uma
estratégia muito superior a qualquer método do tipo \textit{hill climbing}. 

Como pode ser observado, em todos os casos, os melhores indivíduos foram localizados no máximo global
da função, bastando a escolha correta dos parâmetros para que a solução desejada
fosse encontrada com precisão. Não obstante, uma variedade genética proporcional a
probabilidade de mutação foi mantida, dada uma escolha correta de $p_2$ e $p_3$ para cada
problema. Nos casos em que $p_2 = 20\%$, os indivíduos, ao final do processo, se encontravam
espalhados por todo o espaço de busca. 

Vale ressaltar porém que o valor necessário de $p_2$ e $p_3$ para que uma população 
tenha a distribuição desejada depende da função a ser otimizada, como pode ser visto
comparando as Figuras \paref{fig:contour_damped_cossine_mut_20} e \paref{fig:evolution_damped_cossine_mut_20}
com as Figuras \paref{fig:contour_near_gaussians_mut_20} e \paref{fig:evolution_near_gaussians_mut_20}.
Assim, a influência dos parâmetros $p_2$, $p_3$, $e_1$, $e_2$ e $e_3$ no comportamento da população
deve ser estudada em cada caso, afim de extrair do algoritmo o resultado desejado.

Outra vantagem do algoritmo é que os máximos locais também puderam ser encontrados.
Isso pode ser observado especialmente na Figura \paref{fig:evolution_damped_cossine_mut_20},
onde visivelmente há uma concentração de indivíduos em $ f_1(x,y) \approx 0,74 $ e 
$ f_1(x,y) \approx 0,29 $, que correspondem aos dois primeiros máximos locais.
Algo similar pode ser observado na segunda função, nas primeiras gerações da Figura 
\paref{fig:evolution_near_gaussians_mut_20}, com alguma concentração da população em
vermelho e em laranja em $f_1(x,y) \approx 0,8$.

Claro que, mesmo nos casos em que não é possível inferir o máximo local da evolução
de uma parcela da população\trav como ocorre na Figura \paref{fig:evolution_damped_cossine}\trav
uma análise estatística feita sobre os valores das funções desempenho dos indivíduos e
suas respectivas posições ainda seria capaz de acusá-lo.

Em suma, o algoritmo desenvolvido produziu resultados favoráveis nos testes apresentados, e suas capacidades
e limitações serão postos a prova em outros testes subsequentes e nas futuras etapas do trabalho, 
onde será feita sua aplicação no estudo acerca de algum sistema físico.

\nocite{charbonneau2002ga}