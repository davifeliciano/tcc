\chapter{Conclusão}
\label{cap_conclusao}

O algoritmo genético implementado se mostrou eficaz na solução de problemas de
otimização numérica, encontrando estimativas aceitáveis para os máximos globais
das funções utilizadas com poucas iterações, como pode ser observado nas Figuras
\ref{fig:contour_damped_cossine} a \ref{fig:evolution_near_gaussians_mut_20}. Assim, foi
confirmada a vantagem do uso desse tipo de processo sobre métodos do tipo
\textit{hill climbing} na tarefa de mapear não só os máximos globais, mas também
os máximos locais de funções pouco comportadas.

Foi constatado também o impacto do uso de uma maior probabilidade de mutação na
evolução dos indivíduos, ocasionando uma maior distribuição de indivíduos como
ilustrado nas Figuras \ref{fig:contour_damped_cossine_mut_20} e
\ref{fig:contour_near_gaussians_mut_20}. O uso dessa estratégia pode facilitar a
localização dos extremos locais, como mostra a comparação entre as Figuras
\ref{fig:evolution_damped_cossine} e \ref{fig:evolution_damped_cossine_mut_20}.

Ao aplicar o algorítimo para o ajuste dos parâmetros da Hamiltoniana do modelo 
$ k \cdot p $ para os materiais \ch{CrS2} e \ch{CrSe2} foi posta a prova sua
eficácia para a procura em espaços de busca de maior dimensão. O ajuste foi
feito minimizando o desvio quadrático médio entre seus autovalores e as energias
correspondentes calculadas via DFT, uma estratégia que possibilita o uso da
biblioteca desenvolvida para o ajuste de curvas em geral.

Com os parâmetros ajustados, foram calculados os níveis de Landau que surgem nos
vales $K$ e $K'$ como consequência da aplicação de um campo magnético externo
normal à monocamada desses materiais. O resultado obtido coincidiu com os
existentes para o \ch{MoS2} \cite{dias2016tmdc}, o que comprova a precisão do
modelo na aproximação das bandas de condução e de valência na vizinhança desses
vales.

Finalmente, a comparação entre o tempo de execução do algoritmo genético no
ajuste das bandas de energia com o do algoritmo \textit{Dual Annealing} nos
alerta para a possibilidades de melhorias no desempenho da implementação do
algoritmo, uma vez que o último também é implementado em Python, via NumPy, e
apresentou melhor performance. Tendo isso em vista, para problemas de maior
complexidade, o ideal é reescrever a biblioteca em uma linguagem de maior
desempenho, como C ou Fortran, ou mudar sua arquitetura de forma a fazer uso
mais intensivo do NumPy nas operações de maior custo, evitando o uso de listas.

\nocite{ribeiro2013ga}