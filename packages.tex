\documentclass[
  % --- Opções da classe memoir ---
  12pt,                     % Tamanho da fonte
  % openright,              % Capítulos começam em página ímpar (insere página vazia caso preciso)
  oneside,                  % Para impressão em verso e anverso. Oposto a oneside
  a4paper,                  % Tamanho do papel
  % --- opções da classe abntex2 ---
  % chapter=TITLE,          % Títulos de capítulos convertidos em letras maiúsculas
  % section=TITLE,          % Títulos de seções convertidos em letras maiúsculas
  % subsection=TITLE,       % Títulos de subseções convertidos em letras maiúsculas
  % subsubsection=TITLE,    % Títulos de subsubseções convertidos em letras maiúsculas
  % --- opções do pacote babel ---
  % english,                % Idioma adicional para hifenização
  % french,                 % Idioma adicional para hifenização
  % spanish,                % Idioma adicional para hifenização
  portuguese,
  brazil,                    % O último idioma é o principal do documento
  hyperfootnotes=false
]{abntex2}

% ---
% Pacotes básicos
% ---
% \usepackage{fourier}              % Fonte Utopia (serif)
% \usepackage{libertine}            % Fonte Libertine (serif) e Biolinum (sans)
% \usepackage{XCharter}             % Fonte serifada principal do texto: BT Charter
\usepackage[sc]{mathpazo}           % Fonte serifada principal do texto: Palatino Linotype
% \linespread{1.05}                 % Palladio/Palatino needs more leading (space between lines)
\usepackage{biolinum}               % Fonte sem serifa principal
\usepackage[utf8]{inputenc}         % Codificacao do documento
\usepackage[T1]{fontenc}            % Seleção de codigos de fonte
\usepackage{lastpage}               % Usado pela Ficha catalográfica
\usepackage{indentfirst}            % Indenta o primeiro parágrafo de cada seção.
\usepackage{color}                  % Controle das cores
\usepackage{graphicx}               % Inclusão de gráficos
\usepackage{microtype}              % Para melhorias de justificação
\usepackage{amsmath}
\usepackage{mathtools}
\usepackage{nicefrac}
\usepackage{physics}

% Pacote para formulas químicas
\usepackage{chemformula}

% Pacote para unidades do SI
\usepackage{siunitx}
\sisetup{output-decimal-marker = {,}}
\sisetup{group-digits=false}

% Pacote para evitar que figuras sejam renderizadas fora da seção correspondente
\usepackage[section]{placeins}

\usepackage{subcaption}
\captionsetup{subrefformat=parens}

% Fixing up footnotes
\usepackage[symbol]{footmisc}
\renewcommand{\thefootnote}{\fnsymbol{footnote}}

% Pacote para a inclusão de pdfs no documento
\usepackage{pdfpages}

\newtheorem{teo}{Teorema}[chapter]
\newtheorem{exer}{Exercício}
\newtheorem{cor}{Corolário}
\newtheorem{lem}{Lema}
\newtheorem{pro}{Proposição}[chapter]
\newtheorem{exemplo}{Exemplo}[chapter]
\newtheorem{defi}{Definição}[chapter]
\newtheorem{prop}{Propriedade}[chapter]

% ---
% NOVOS COMANDOS
% ---
\newcommand{\N}{\mathbb{N}}
\newcommand{\Z}{\mathbb{Z}}
\newcommand{\R}{\mathbb{R}}
\newcommand{\Q}{\mathbb{Q}}
\newcommand{\C}{\mathbb{C}}
\newcommand{\trav}{\,---\,}
\newcommand{\mathperiod}{\;\mathrm{.}}
\newcommand{\mathcomma}{\;\mathrm{,}}
\newcommand{\mathdots}{\;\cdots\;}
\newcommand{\bvec}[1]{\mathbf{#1}}


% ---
% Pacotes adicionais
% ---
\usepackage[ruled,vlined,lined,linesnumbered,algochapter,portuguese]{algorithm2e}
\usepackage{listings}
\usepackage{multicol}
\usepackage{xspace}
\newcommand{\supress}{{[}\ldots{]}\xspace}

% ---
% Pacotes de citações
% ---
% \usepackage[brazilian,hyperpageref]{backref}        % Paginas com as citações na bibl
\usepackage[num]{abntex2cite}                         % Citações padrão ABNT
\citebrackets[]

% ---
% CONFIGURAÇÕES DE PACOTES
% ---

% ---
% Configurações do pacote backref
% Usado sem a opção hyperpageref de backref
% \renewcommand{\backrefpagesname}{Citado na(s) página(s):~}
% Texto padrão antes do número das páginas
% \renewcommand{\backref}{}
% Define os textos da citação
% \renewcommand*{\backrefalt}[4]{
% \ifcase #1 %
%   Nenhuma citação no texto.%
% \or
%   Citado na página #2.%
% \else
%   Citado #1 vezes nas páginas #2.%
% \fi}%


% ---
% Configurações de aparência do PDF final
% ---

% Alterando o aspecto da cor azul
\definecolor{blue}{RGB}{41,5,195}

% Informações do PDF
\makeatletter
\hypersetup{
  % pagebackref=true,
  pdftitle={\@title},
  pdfauthor={\@author},
  pdfsubject={\imprimirpreambulo},
  pdfcreator={LaTeX with abnTeX2},
  pdfkeywords={Algoritmo Genético}{Otimização Numérica}{Máximos de Funções}{Mínimos de Funções}{Python},
  colorlinks=true,           % false: boxed links; true: colored links
  linkcolor=blue,            % color of internal links
  citecolor=blue,            % color of links to bibliography
  filecolor=magenta,         % color of file links
  urlcolor=blue,
  bookmarksdepth=4,
  hidelinks
}
\makeatother

% ---
% Espaçamentos entre linhas e parágrafos
% ---
% O tamanho do parágrafo é dado por:
\setlength{\parindent}{1.3cm}

% Controle do espaçamento entre um parágrafo e outro:
\setlength{\parskip}{0.2cm}  % tente também \onelineskip

% ---
% Apresentação/formatação da epígrafe dos capítulos
% ---
\epigraphfontsize{\small\itshape}
\setlength{\epigraphwidth}{0.6\textwidth}
\setlength{\epigraphrule}{0pt}

\usepackage{xifthen} % provides \isempty command below
\newcommand{\epiauthor}[2][]{%
  \ifthenelse{\isempty{#1}}%
  {\textnormal{\textsc{---#2}}}% if #1 is empty
  {\textnormal{\textsc{---#2}}, #1}% if #1 is not empty
}

\usepackage{tikz}
\usetikzlibrary{arrows.meta, positioning}
\usetikzlibrary{decorations.markings}
\usetikzlibrary{shapes.geometric,shapes.symbols,shapes.misc}
\tikzset{
  start_end/.style={
      draw,
      rectangle,
      rounded corners,
      text width=3cm,
      minimum height=1.3cm,
      text centered,
    },
  process/.style={
      draw,
      text width=3cm,
      minimum height=1.3cm,
      text centered,
    },
  decision/.style={
      draw,
      diamond,
      aspect=2,
      text width=3cm,
      minimum height=1.3cm,
      text centered,
    },
  io/.style={
      draw,
      trapezium,
      trapezium left angle=70,
      trapezium right angle=110,
      text width=3cm,
      minimum height=1.3cm,
      text centered,
    },
  document/.style={
      draw,
      tape,
      tape bend top=none,
      text width=3cm,
      minimum height=1.3cm,
      text centered,
    },
  description/.style={
      text centered,
      text width=5cm,
    },
  myarrow/.style={
      postaction={
          decorate, decoration={
              markings,mark=at position #1 with {\arrow{Stealth};}
            }
        }
    },
}

\usetikzlibrary{chains,fit,calc,patterns}
\edef\sizetape{0.7cm}
\tikzset{
  gene/.style={
      draw,
      minimum size=\sizetape,
    },
  gene_filled/.style={
      draw,
      fill=black!35,
      minimum size=\sizetape,
    },
  gene_mutated/.style={
      draw,
      pattern=north west lines,
      pattern color=black!25,
      minimum size=\sizetape,
    },
}