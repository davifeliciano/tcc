Nas diversas áreas do conhecimento humano é bem conhecida uma série de problemas de otimização. 
A solução desses problemas consiste em encontrar em um conjunto de configurações, um subconjunto,
ou mesmo um elemento que melhor satisfaça um ou mais vínculos previamente determinados. Uma
estratégia famosa por solucionar problemas de tal classe de forma rápida e eficiente consiste no
emprego dos algoritmos genéticos. São assim chamados devido à forte inspiração em fenômenos
da biologia evolutiva\trav como mutação, recombinação (ou \textit{crossover}) e seleção\trav
na elaboração de suas etapas de execução. Neste trabalho é proposto um algoritmo genético para
a otimização de funções reais, isto é, para a procura dos pontos nos quais a função é máxima
(ou mínima). Sua implementação é feita em Python com o uso da biblioteca NumPy. 
O algoritmo é aplicado na otimização de algumas funções de exemplo, e é feita uma discussão acerca dos
resultados, e da eficácia e performance do programa. 

\vspace{\onelineskip}\noindent
\textbf{Palavras-chaves}: Algoritmo Genético. Otimização Numérica. Python. NumPy.